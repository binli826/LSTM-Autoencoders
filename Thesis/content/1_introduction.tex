\chapter{Introduction}
\label{chap:Introduction}

Anomaly detection is an important problem in data mining, and widely used in the manufacturing industry, commercial world, internet company etc. It could avoid or reduce lose in many scenarios like machine health monitoring, credit card fraud detecting and spam email classification, and could also be used as a preprocessing step to remove anomalies for datasets. There are already plenty of anomaly detection techniques proposed in previous literatures, that solve this problem from variety perspectives, e.g. distance-based methods, clustering analysis, density-based methods etc.\\

There is no lack of anomaly detection approaches that perform good with respect to different kinds of data, however, majority of them are batch model, which means, all data should be available in advance. This becomes a shortcoming under today’s big data background. With the rapid development of hardware in the last decade, the situation of data acquisition and analysis has significantly been changed. Specifically, the IoT application. Assume that we collect data from sensors attached to IoT devices, the data comes continuously and everlasting. During data analysis, we should always consider the volume and velocity of data, which means, on one hand, with traditional batch classifiers, the infinity data stream will lead to out of memory, on the other hand, streaming data usually comes with a high speed that leaving the system few processing time. In addition, the statistical property of data may also change over time, which is formally called ‘concept drift’. The model should always learn new knowledge from the stream and update its definition of normal and anomalous automatically. To this end, an anomaly detection system for streaming data should be able to 1) be initialized with only a small subset, 2) process streaming data and make prediction in real-time, 3) adapt data evolution over time.\\

Malhotra et al. introduced an autoencoder based anomaly detection approaches in [1],[2], and achieved good performance in multiple time series dataset. However, in this approach, they assume that the whole datasets are available beforehand, and didn’t considered the aforementioned online learning difficulties. Hence, we enhanced this kind of autoencoder based anomaly detection approaches with the online learning ability by inplementing incremental model updating strategies based on the streaming data.\\

In this paper, we introduce a novel and robust incremental autoencoder-based anomaly detection model, which designed specifically for time series data in a streaming fashion using Long Short-Term memory (LSTM) units, with also online learning ability for model updating.  For each accumulated mini-batch of streaming data, the autoencoder reconstructs it with previous knowledge learned from normal data. Anomaly data (never used for training) is expected to cause significant larger reconstruction error than normal data. In addition, the model update itself online according to performance-based criterions.