\chapter{Experimental results}
\label{chap:results}


\section{Anomaly detection performance}
\label{sec:performance}

With parameters learned from \Fref{sec:parametertuning}, autoencoders are learned for each dataset with the beginning of streaming data. The anomaly detection performance is described by AUC. For each dataset, we compare the AUC of online phase that without and with continuously model and parameter retraining (\Fref{tab:performance}). 

\begin{table}[h] 
\caption{Performance} 
\centering      
\begin{tabular}{c | c | c | c}  
\hline  
Dataset & AUC(without retraining) & AUC(with retraining) & \#retrain \\ 
\hline 
PowerDemand & 0.91 & 0.97 & 1  \\  
\hline 
SMTP &  &  &  \\ 
\hline 
SMTP+HTTP &  &  & \\ 
\hline 
HTTP &  &   &   \\ 
\hline 
ForestCover & &  & \\   
\hline    
\end{tabular}
\label{tab:performance}  
\end{table} 

\section{Synthetic example}
\label{sec:synthetic}

In order to show the benefit of model retraining along the stream, we demonstrate the online learning process of the small set Power demand in this section. The power demand dataset does not contain clear incremental or sudden concept drift, but the normal pattern still different slightly to each other. Lack of overall impression during the model initialization phase can lead to failures in the online phase. \Fref{fig:power_retraining} shows 3 continual days power demand in normal state. Due to the lack knowledge of current pattern, the autoencoder reconstructs the input time series high than desired. A model retraining process is triggered after the second day with last seen data, and the model performs well again on the third day.

\begin{figure}[h]
\centering
\includegraphics[width=15cm, height=4cm]{power_retraining}
\caption[Retraining effect on Power Demand dataset]{Retraining effect on Power Demand dataset}
\label{fig:power_retraining}
\end{figure}

During the online phase, the model is retrained two times, before batch No.10 and No. 27. After retraining, the normal data reconstruction error becomes lower while for abnormal data becomes higher, so that the classification becomes easier.

\begin{figure}[h]
\centering
\includegraphics[width=10cm, height=4cm]{power_online_score}
\caption[Power Demand dataset online learning scores]{Power Demand dataset online learning scores}
\label{fig:power_online}
\end{figure}

After each retraining process, the parameters mu, sigma and threshold of anomaly scores are also updated. \Fref{fig:parachanges} shows the parameter changes over the stream. As there is no clear concept drift during the power demand stream, the parameters changes just slightly, and learn latest knowledge from the retrain buffer. 

\begin{figure}[h]
\centering
\includegraphics[width=6cm, height=4cm]{para_update}
\caption[Online parameter updating]{Online parameter updating}
\label{fig:parachanges}
\end{figure}

During the online phase, each normal window that is not given with a low enough anomaly score is appended to the retaining buffer to accumulate the retaining set. In order to find out what kind of data is used for retaining and how much retraining data is enough for model updating, we experiment with different retraining buffer size on the power demand stream.




\section{Model retraining}
\label{sec:retraining}

\subsection{Reaction of concept drift and distribution changes}
\label{sec:reaction}
A main advantage of online model is its ability to take reaction against sudden data distributional changes in time. The SMTP+HTTP data set is composed by directly connect HTTP set after SMTP, so there is a sudden concept drift in between. The model is initialized with only SMTP data, so HTTP is completely new knowledge.\n


Another experiment on concept drift is to detect, how the different kinds of forest cover types (type 1 to 7) in the FOREST dataset, which are all treated as normal state except type 4, are relative to the model updating. The FOREST model is trained with hidden size 45 and window length 20. In the beginning, 3000 windows are used for initialization, and 26050 windows comes as stream.  Every normal window contains more than 10 scores over threshold is treated as hard window and appended to retraining buffer. Also, every abnormal window is saved for threshold updating.  When retrain buffer size reaches 750, a retraining process will be triggered. The model is retrained 3 times over stream.

\begin{figure}[h]
\centering
	\begin {subfigure}[t]{0.45\textwidth}
	\centering
	\includegraphics[height=4cm]{pie1}
	\caption{Initialization dataset}
	\label{fig:init}
	\end{subfigure}
	~
	\begin {subfigure}[t]{0.45\textwidth}
	\centering
	\includegraphics[height=4cm]{pie2}
	\caption{Retrain Buffer 1}
	\label{fig:buf1}
	\end{subfigure}
	~
	\begin {subfigure}[t]{0.45\textwidth}
	\centering
	\includegraphics[ height=4cm]{pie3}
	\caption{Retrain Buffer 2}
	\label{fig:buf2}
	\end{subfigure}
	~
	\begin {subfigure}[t]{0.45\textwidth}
	\centering
	\includegraphics[height=4cm]{pie4}
	\caption{Retrain buffer 3}
	\label{fig:buf3}
	\end{subfigure}

	\caption[FOREST data initializaiton and retrain buffer data distribution]{FOREST data initializaiton and retrain buffer data distribution: N1 to N6 represent normal states from the 6 different cover type classes, and A represents anomaly, namely type4. (a) is the initialization data distribution and (b), (c), (d) are the three retrain buffers' data distribution. Outer rings are all seen data since last retraining and inner rings are the portions added to retrain buffer.}
\label{fig:bufs}
\end{figure}








